
    
    
    
    


    


    
    \section{jupyter\_contrib\_nbextensions}\label{jupyterux5fcontribux5fnbextensions}

This PyMOTW is about
\href{https://github.com/ipython-contrib/jupyter_contrib_nbextensions}{jupyter\_contrib\_nbextensions},
a repository containing a collection of extensions that add
functionality to the Jupyter notebook. These extensions are mostly
written in Javascript and will be loaded locally in your browser.

LINKS: *
https://github.com/ipython-contrib/jupyter\_contrib\_nbextensions *
http://jupyter-contrib-nbextensions.readthedocs.io

The notebook extensions can be selected and configured in your Jupyter
home tree via the tab \emph{Nbextensions}
(http://localhost:8888/tree\#nbextensions\_configurator).

    \subsection{Table of contents 2 (toc2)}\label{table-of-contents-2-toc2}

The toc2 extension enables to collect all running headers and display
them in a floating window, as a sidebar or with a navigation menu. The
extension is also draggable, resizable, collapsable, dockable and
features automatic numerotation with unique links ids, and an optional
toc cell.

Have a look at the options in the Nbextensions tab.

\subsubsection{Conclusion}\label{conclusion}

\textbf{I love this}. Simple and easy. Look at the sidebar table of
contents. Neat, right? Let's you jump to different sections quite easy
and also shows you which section is {currently selected} and which
section is {queued to run}.

    \subsection{LaTeX\_envs}\label{latexux5fenvs}

\subsubsection{\texorpdfstring{Equations,
\texttt{\textbackslash{}ref\{eq:\}}, theorem
environments}{Equations, \textbackslash{}ref\{eq:\}, theorem environments}}\label{equations-refeq-theorem-environments}

The following code
The dot-product is defined by equation (\ref{eq:dotp}) in theorem \ref{theo:dotp} just below:
\begin{theorem}[Dot Product] \label{theo:dotp}
Let $u$ and $v$ be two vectors of $\mathbb{R}^n$. The dot product can be expressed as
\begin{equation}
\label{eq:dotp}
u^Tv = |u||v| \cos \theta,
\end{equation}
where $\theta$ is the angle between $u$ and $v$ ...
\end{theorem}
    renders like this:

The dot-product is defined by equation (\ref{eq:dotp}) in theorem
\ref{theo:dotp} just below: \begin{theorem}[Dot Product]
\label{theo:dotp} Let \(u\) and \(v\) be two vectors of
\(\mathbb{R}^n\). The dot product can be expressed as

\begin{equation}
\label{eq:dotp}
u^Tv = |u||v| \cos \theta,
\end{equation}

where \(\theta\) is the angle between \(u\) and \(v\) ...
\end{theorem}

    \subsubsection{Citations, bibliography and
references}\label{citations-bibliography-and-references}

It seems, that you can place a \texttt{.bib} file in the same directory
as the notebook and use \texttt{\textbackslash{}cite\{\}} to generate
citations. Let's see if it works: \cite{Thomson1887}, \cite{Hebb1949}

Unfortunately, it does not really work for me, maybe because
nbextensions said that latex\_envs was \emph{possibly incompatible}. My
output after clicking \emph{Read bibliography and generate references
section} is the following:

    \paragraph{References}\label{references}

{[}Thomson1887{]} !! \emph{This reference was not found in library.bib }
!!

{[}Hebb1949{]} !! \emph{This reference was not found in library.bib } !!

    \subsubsection{Conclusions}\label{conclusions}

Probably awesome if it works. Especially citations would be great to
have. However, I don't know if I really need the environments. Math
works alright in markdown also without this extension.

    \subsection{Exercise and Exercise2}\label{exercise-and-exercise2}

These are two extensions for Jupyter, for hiding/showing solutions
cells.

\subsubsection{Exercise}\label{exercise}

\begin{exercise}[Laplace Equation] Write the Laplace
equation \(\Delta\Phi=0\) in spherical coordinates.
\end{exercise}

    SOLUTION

    \begin{equation}
\frac{1}{r^2}\frac{\partial}{\partial r} \left( r^2 \frac{\partial}{\partial r} \Phi \right) + \frac{1}{r^2 \sin\theta}\frac{\partial}{\partial \theta} \left( \sin\theta \frac{\partial}{\partial \theta} \Phi \right) + \frac{1}{r^2 \sin^2\theta}\frac{\partial}{\partial \phi^2} \Phi
\end{equation}

    \subsubsection{Exercise2}\label{exercise2}

\begin{exercise}[Laplace Equation] Write the Laplace
equation \(\Delta\Phi=0\) in spherical coordinates.
\end{exercise}

    SOLUTION

    \begin{equation}
\frac{1}{r^2}\frac{\partial}{\partial r} \left( r^2 \frac{\partial}{\partial r} \Phi \right) + \frac{1}{r^2 \sin\theta}\frac{\partial}{\partial \theta} \left( \sin\theta \frac{\partial}{\partial \theta} \Phi \right) + \frac{1}{r^2 \sin^2\theta}\frac{\partial}{\partial \phi^2} \Phi
\end{equation}

    \subsubsection{Conclusions}\label{conclusions}

I encounter some bugs with Exercise2, the solution shows up after
reloading the notebook. Therefore the winner is Exercise! Could be
useful for workshops..

    \subsection{Python in markdown}\label{python-in-markdown}
%
\begin{lstlisting}
a=3.765
\end{lstlisting}
    Print the value of the variable in markdown using
a={{a}}
    a=3.765

    \subsubsection{Conclusion}\label{conclusion}

\texttt{¯\textbackslash{}\_(ツ)\_/¯}

    \section{Interactive and animated
plots}\label{interactive-and-animated-plots}

\subsection{Some random data to plot}\label{some-random-data-to-plot}
%
\begin{lstlisting}
import numpy as np
import elephant
import neo
import quantities as pq
from sklearn.decomposition import PCA
import matplotlib.pyplot as plt
from mpl_toolkits.mplot3d import Axes3D
from IPython.display import HTML
import matplotlib.animation as animation
\end{lstlisting}%
\begin{lstlisting}
# Load block from ANDA data
block = np.load('/home/papen/git_repos/ANDA2017/data/data2.npy').item()

# Get spiketrain of first regular trial
sts = []
idx = block.annotations['all_trial_ids']
sts.append(block.filter(targdict={'trial_id': idx[0]}, objects=neo.Segment)[0].spiketrains)

# Generate binned time series of spike counts and apply PCA
binsize = 100*pq.ms
binned = elephant.conversion.BinnedSpikeTrain(sts[0], binsize=binsize).to_array()
pca = PCA(n_components=3)
pca.fit(binned.T)
PC  = np.matmul(pca.components_,binned)[:3,:]
t = np.arange(len(PC[i,:]))*binsize
\end{lstlisting}
    \subsection{Matplotlib notebook}\label{matplotlib-notebook}

Using \texttt{\%matplotlib\ notebook} activates the nbagg backend added
in matplotlib 1.4, which will include a javascript interface for
interaction with inline figures in the notebook. Apparently for Python2
one can also use \texttt{\%matplotlib\ nbagg}.
%
\begin{lstlisting}
%matplotlib notebook
fig = plt.figure(figsize=(10,5))
ax = fig.add_subplot(121, projection='3d')
ax.plot(PC[0,:], PC[1,:], PC[2,:], '-k')
ax.set_xlabel('Comp. 1')
ax.set_ylabel('Comp. 2')
ax.set_zlabel('Comp. 3')

ax = fig.add_subplot(122)
for i in xrange(len(PC[:,0])):
    ax.plot(t, PC[i,:], '-', label='PC {}'.format(i))
ax.set_xlabel('time [ms]')
ax.set_ylabel('PC')
plt.legend()
\end{lstlisting}%
    
    

    
    
    

    
    
    
    

    

    \subsection{Animated plots with HTML}\label{animated-plots-with-html}

    The following example is taken from
https://matplotlib.org/examples/animation/simple\_anim.html
%
\begin{lstlisting}
%matplotlib inline

fig = plt.figure()
ax = fig.add_subplot(111)

line, = ax.plot(t, binned[0,:])

def animate(i, title_num=None):
    line.set_ydata(binned[i,:])  # update the data
    ax = plt.gca()
    ax.set_title('Spike counts of unit {}'.format(i))
    return line,

# Init only required for blitting to give a clean slate.
def init():
    line.set_ydata([])
    return line,

ani = animation.FuncAnimation(fig, animate, np.arange(1, len(binned[:,0])), init_func=init,
                              interval=500, blit=True)
ax1.set_xlabel('time [ms]')
ax1.set_ylabel('spike count')

HTML(ani.to_html5_video())
\end{lstlisting}
    \section{Nbconvert}\label{nbconvert}
%
\begin{lstlisting}
import elephant
from elephant.spike_train_generation import homogeneous_poisson_process as HPP
from quantities import Hz, ms
\end{lstlisting}%
\begin{lstlisting}
data1 = [HPP(rate=10*Hz, t_start=0*ms, t_stop=60000*ms) for _ in range(200)]
data2 = [HPP(rate=10.2*Hz, t_start=0*ms, t_stop=60000*ms) for _ in range(200)]
\end{lstlisting}%
\begin{lstlisting}
w = 200*ms

# Compute PSTH
psth1 = elephant.statistics.time_histogram(data1, binsize = w, output='rate').rescale('Hz')
psth2 = elephant.statistics.time_histogram(data2, binsize = w, output='rate').rescale('Hz')

# Compute ISI dist
ISIpdf1 = np.histogram(elephant.statistics.isi(data1), bins=w, rng=[0*pq.s, None], density=True)
ISIpdf2 = misc.isi_pdf(data2, bins=w, rng=[0*pq.s, None], density=True)
\end{lstlisting}%
    \begin{Verbatim}[commandchars=\\\{\}]

        

        NameErrorTraceback (most recent call last)

        <ipython-input-10-9c3f52b8fad6> in <module>()
          6 
          7 \# Compute ISI dist
    ----> 8 ISIpdf1 = misc.isi\_pdf(data1, bins=w, rng=[0*pq.s, None], density=True)
          9 ISIpdf2 = misc.isi\_pdf(data2, bins=w, rng=[0*pq.s, None], density=True)


        NameError: name 'misc' is not defined

    \end{Verbatim}
%
\begin{lstlisting}
# Set figure
plt.figure(figsize=(12,8))
plt.subplots_adjust(top=.85, right=.85, left=.2, bottom=.1, hspace=.7 , wspace=.3)
num_row,num_col = 3, 4

# Raster plots
ax = plt.subplot2grid((num_row,num_col), (0, 0), rowspan=1, colspan=3)
misc.raster(data1, ms = .5, xlabel='time', ylabel='unit id', color='k')
plt.title('data 1')
plt.subplot2grid((num_row,num_col), (1, 0), rowspan=1, colspan=3, sharex=ax)
misc.raster(data2, ms = .5, xlabel='time', ylabel='unit id', color='g')
plt.title('data 2')

# PSTH
plt.subplot2grid((num_row,num_col), (2, 0), rowspan=1, colspan=3,sharex=ax)
plt.title('PSTH')
plt.ylabel('# spikes')
plt.plot(psth1.times.rescale('ms'), psth1,color='k')
plt.plot(psth2.times.rescale('ms'), psth2,color='g')
ax.set_xlim(0, max(psth1.times.rescale('ms').magnitude))

# Spike counts per trial
plt.subplot2grid((num_row,num_col),(0, 3), rowspan=1, colspan=1)
plt.plot([len(st) for st in data1],range(1, len(data1)+1), '.-', ms=5, color='k')
plt.title('Spike Count')
plt.subplot2grid((num_row,num_col),(1, 3), rowspan=1, colspan=1)
plt.plot([len(st) for st in data2],range(1,len(data2)+1), '.-', ms=5, color='g')

# ISI distribution
plt.subplot2grid((num_row,num_col),(2, 3), rowspan = 2, colspan = 1)
plt.plot(ISIpdf1.times,ISIpdf1,color = 'k')
plt.plot(ISIpdf2.times,ISIpdf2,color = 'g')
plt.yscale('log')
plt.ylim(1e-4, 0.1)
plt.xlim(0, 20)
plt.title('ISI distribution')

plt.show()
\end{lstlisting}%
    \begin{Verbatim}[commandchars=\\\{\}]

        

        NameErrorTraceback (most recent call last)

        <ipython-input-11-01d3e12d994b> in <module>()
          1 \# Set figure
    ----> 2 plt.figure(figsize=(12,8))
          3 plt.subplots\_adjust(top=.85, right=.85, left=.2, bottom=.1, hspace=.7 , wspace=.3)
          4 num\_row,num\_col = 3, 4
          5 


        NameError: name 'plt' is not defined

    \end{Verbatim}


    % Add a bibliography block to the postdoc
    
    
%\bibliographystyle{ieetran}
%\bibliography{Thesis}

    
    